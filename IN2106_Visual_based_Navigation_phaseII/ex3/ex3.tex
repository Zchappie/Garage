\documentclass[12pt,letterpaper]{article}
\usepackage{fullpage}
\usepackage[top=2cm, bottom=4.5cm, left=2.5cm, right=2.5cm]{geometry}
\usepackage{amsmath,amsthm,amsfonts,amssymb,amscd}
\usepackage{lastpage}
\usepackage{enumerate}
\usepackage{fancyhdr}
\usepackage{mathrsfs}
\usepackage{xcolor}
\usepackage{graphicx}
\usepackage{listings}
\usepackage{hyperref}

\hypersetup{%
  colorlinks=true,
  linkcolor=blue,
  linkbordercolor={0 0 1}
}
 
\renewcommand\lstlistingname{Algorithm}
\renewcommand\lstlistlistingname{Algorithms}
\def\lstlistingautorefname{Alg.}

\lstdefinestyle{Python}{
    language        = Python,
    frame           = lines, 
    basicstyle      = \footnotesize,
    keywordstyle    = \color{blue},
    stringstyle     = \color{green},
    commentstyle    = \color{red}\ttfamily
}

\setlength{\parindent}{0.0in}
\setlength{\parskip}{0.05in}

% Edit these as appropriate
\newcommand\course{VisNav}
\newcommand\hwnumber{3}                  % <-- homework number
\newcommand\NetIDa{Meng Liu}           % <-- NetID of person #1
%\newcommand\NetIDb{netid12038}           % <-- NetID of person #2 (Comment this line out for problem sets)

\pagestyle{fancyplain}
\headheight 35pt
\lhead{\NetIDa}
%\lhead{\NetIDa\\\NetIDb}                 % <-- Comment this line out for problem sets (make sure you are person #1)
\chead{\textbf{\Large Exercise \hwnumber}}
\rhead{\course \\ \today}
\lfoot{}
\cfoot{}
\rfoot{\small\thepage}
\headsep 1.5em

\begin{document}

\section*{Exercise 2: Epipolar constraint}
According to the stereo vision geometry, we can get the following relationship:
\begin{equation*}
	\begin{split}
		(R\cdot XO_R + O_RO_L)\times XO_L &=0\\
		(R\cdot XO_R)\times XO_L + O_RO_L\times XO_L &=0\\
		(R\cdot XO_R)\times XO_L + O_RO_L\times (R\cdot XO_R +O_RO_L) &=0\\
		(R\cdot XO_R)\times XO_L + O_RO_L\times (R\cdot XO_R) &=0\\
		XO_L^T\cdot (R\cdot XO_R)\times XO_L + XO_L^T\cdot (O_RO_L\times (R\cdot XO_R)) &=0\\
		XO_L^T\cdot (O_RO_L\times (R\cdot XO_R)) &=0\\
		XO_L^T\cdot [O_RO_L]_{\times} R\cdot XO_R &=0\\
		XO_L^T\cdot E\cdot XO_R &=0
	\end{split}
\end{equation*}
Thus the essential matrix is in form $E= [O_RO_L]_{\times} R $

\end{document}