%%% Template originaly created by Karol Kozioł (mail@karol-koziol.net) and modified for ShareLaTeX use

\documentclass[a4paper,11pt]{article}

\usepackage{comment}

\usepackage[T1]{fontenc}
\usepackage[utf8]{inputenc}
\usepackage{graphicx}
\usepackage{xcolor}

\renewcommand\familydefault{\sfdefault}
\usepackage{tgheros}
\usepackage[defaultmono]{droidmono}

\usepackage{amsmath,amssymb,amsthm,textcomp}
\usepackage{enumerate}
\usepackage{multicol}
\usepackage{tikz}

\usepackage{geometry}
\geometry{total={210mm,297mm},
left=25mm,right=25mm,%
bindingoffset=0mm, top=20mm,bottom=20mm}


\linespread{1.3}

\newcommand{\linia}{\rule{\linewidth}{0.5pt}}

% custom theorems if needed
\newtheoremstyle{mytheor}
    {1ex}{1ex}{\normalfont}{0pt}{\scshape}{.}{1ex}
    {{\thmname{#1 }}{\thmnumber{#2}}{\thmnote{ (#3)}}}

\theoremstyle{mytheor}
\newtheorem{defi}{Definition}

% my own titles
\makeatletter
\renewcommand{\maketitle}{
\begin{center}
\vspace{2ex}
{\huge \textsc{\@title}}
\vspace{1ex}
\\
\linia\\
\@author \hfill \@date
\vspace{4ex}
\end{center}
}
\makeatother
%%%

% custom footers and headers
\usepackage{fancyhdr}
\pagestyle{fancy}
\lhead{}
\chead{}
\rhead{}
\lfoot{Assignment \textnumero{} 1}
\cfoot{}
\rfoot{Page \thepage}
\renewcommand{\headrulewidth}{0pt}
\renewcommand{\footrulewidth}{0pt}
%

% code listing settings
\usepackage{listings}
\lstset{
    language=Python,
    basicstyle=\ttfamily\small,
    aboveskip={1.0\baselineskip},
    belowskip={1.0\baselineskip},
    columns=fixed,
    extendedchars=true,
    breaklines=true,
    tabsize=4,
    prebreak=\raisebox{0ex}[0ex][0ex]{\ensuremath{\hookleftarrow}},
    frame=lines,
    showtabs=false,
    showspaces=false,
    showstringspaces=false,
    keywordstyle=\color[rgb]{0.627,0.126,0.941},
    commentstyle=\color[rgb]{0.133,0.545,0.133},
    stringstyle=\color[rgb]{01,0,0},
    numbers=left,
    numberstyle=\small,
    stepnumber=1,
    numbersep=10pt,
    captionpos=t,
    escapeinside={\%*}{*)}
}

%%%----------%%%----------%%%----------%%%----------%%%

\begin{document}

\title{ACEs Assignment \textnumero{} 2}

\author{Bremen University}

\date{16/04/2018}

\maketitle

\section*{Exercise 1: Circuit Equivalence}

%whole equation thing is blocked
%-------------------------------------------------------------------------------------
\begin{comment}
First, use the logical operations to simplify the equation of both circuits.\\
Left one:
\begin{equation} \label{eq1}
\begin{split}
O & = \neg A \land (B \land C) \lor \neg B \\
 & = (\neg B \lor B) \land (\neg B \lor (\neg A \land C))\\
 & = (\neg B \lor B) \land (\neg B \lor \neg A) \land (\neg B \lor C)
\end{split}
\end{equation}
\\
Right one:
\begin{equation} \label{eq2}
\begin{split}
O'& = (A'\land B') \oplus (\neg (B' \lor C') \lor C' \lor A')\\
& = (A'\land B') \oplus (\neg B' \land \neg C' \lor C' \lor A')\\
& = (\neg A' \lor \neg B')\land ((\neg B' \land \neg C') \lor A' \lor C') \lor (A' \land B')\land (\neg (\neg B' \lor \neg C')\land \neg A' \land \neg C')\\
& = \neg B' \lor (\neg A' \land \neg C') \lor (A' \land \neg B')\\
& = \neg B' \lor (\neg A' \land \neg C')\\
& = (\neg B' \lor \neg A')\land (\neg B' \lor \neg C')
\end{split}
\end{equation}
\\
Verify the equivalency:
\begin{equation} \label{eq3}
(\neg O \lor \neg O')\land (O \lor O')
\end{equation}
\end{comment}
%-------------------------------------------------------------------------------------

% code
\begin{lstlisting}[label={list:first},caption=DIMACS.]
#assignment 2-1
#x, y are the inputs; z is output of the gate
def g_or(x, y, z):
    print(x, y, -z, 0)
    print(-x, z, 0)
    print(-y,z, 0)
def g_and(x, y, z):
    print(-x, -y, z, 0)
    print(x, -z, 0)
    print(y, -z, 0)
def g_not(x, z):
    print(-x, -z, 0)
    print(x, z, 0)
def g_xor(x, y, z):
    print(-x, -y, -z, 0)
    print(x, y, -z, 0)
    print(x, -y, z, 0)
    print(-x, y, z, 0)

#left one
#a, b, c, g1, g2, g3, g4, o = [1,2,3,4,5,6,7,8]
print("p cnf 14 33")    
g_not(1, 4)
g_not(2, 5)
g_and(2, 3, 6)
g_and(4, 6, 7)
g_or(5, 7, 8)    

#right one
#G1,G2,G3,G4,G5,O = [11,12,13,14,15,16]
g_and(1, 2, 11)
g_or(2, 3, 12)
g_or(1, 3, 13)
g_not(12, 14)
g_or(14, 12, 14)
g_xor(11, 14, 15)

#verify the equivlency
print(-8, -16, 0)
print(8, 16, 0)

\end{lstlisting}

\begin{lstlisting}[label={list:second}, caption=CNF]
p cnf 14 33
-1 -4 0
1 4 0
-2 -5 0
2 5 0
-2 -3 6 0
2 -6 0
3 -6 0
-4 -6 7 0
4 -7 0
6 -7 0
5 7 -8 0
-5 8 0
-7 8 0
-1 -2 11 0
1 -11 0
2 -11 0
2 3 -12 0
-2 12 0
-3 12 0
1 3 -13 0
-1 13 0
-3 13 0
-12 -14 0
12 14 0
14 12 -14 0
-14 14 0
-12 14 0
-11 -14 -15 0
11 14 -15 0
11 -14 15 0
-11 14 15 0
-8 -16 0
8 16 0
\end{lstlisting}

SATISFIABLE\\
v -1 -2 -3 4 5 -6 -7 8 -9 -10 -11 -12 -13 14 15 -16 0\\
Since the code verify the no-equivalence, this means the two circuits are not equivalent.

\section*{Problem 2:  DPLL without unit propagation}
Variables $V = \{v_1,v_2...v_n\}$ in Boolean formula (general form):
%\begin{equation*}
\begin{multline*}
    \psi(v_1,v_2...v_n) = (v_1\lor v_2\lor v_3...\lor v_n)\land (v_2\lor v_3...\lor v_n)...\land v_n\land(v_n-1 \lor \neg v_n)\\
    \land (v_n-2\lor \neg v_n-1\lor \neg v_n)...	\land (v_1\lor \neg v_2...\lor \neg v_n)
\end{multline*}
%\end{equation*}
If use the order$ \{v_1,v_2...v_n\}$, the Algorithm 1 will generate
$$\sum_{i=1}^{n} 2^{i} $$
nodes to search the correct answer. But if use the order $\{v_n, v_n-1...v_2, v_1\}$, it will only use 2*n nodes to complete. For $ n\geqslant 5$, the difference of the speed will be clear to see. 

\end{document}
