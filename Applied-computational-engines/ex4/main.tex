\documentclass{base}
% Dateikodierung ist utf8
\usepackage[utf8]{inputenc}
\usepackage{url}
\usepackage[export]{adjustbox}
\usepackage{amsmath}
\usepackage{listings}
\usepackage{tikz}
\usepackage{tabularx}
\usepackage{color,colortbl}
\usepackage{ulem}
\usepackage{pdfpages}
\usepackage{ wasysym }
\usepackage{ booktabs }
\usepackage{lscape}
\usepackage{multicol}
\usepackage{longtable}

\begin{document}

\Abgabeblatt{Assignment 4}{14.5.2018}{????}{????}{Yannis Rohloff (yannis@uni-bremen.de)}{Meng Liu(lium@uni-bremen.de)}{}

\lstset{
    language=Python,
    basicstyle=\ttfamily\small,
    aboveskip={1.0\baselineskip},
    belowskip={1.0\baselineskip},
    columns=fixed,
    extendedchars=true,
    breaklines=true,
    tabsize=4,
    prebreak=\raisebox{0ex}[0ex][0ex]{\ensuremath{\hookleftarrow}},
    frame=lines,
    showtabs=false,
    showspaces=false,
    showstringspaces=false,
    keywordstyle=\color[rgb]{0.627,0.126,0.941},
    commentstyle=\color[rgb]{0.133,0.545,0.133},
    stringstyle=\color[rgb]{01,0,0},
    numbers=left,
    numberstyle=\small,
    stepnumber=1,
    numbersep=10pt,
    captionpos=t,
    escapeinside={\%*}{*)}
}


\section*{Exercise 1: Optimal Propagation}
The following table lists all partial valuations that can be completed to be a fulfilling valuation for the given cnf.
We list all variables that are obtainable from unit propagation with the partial valuation.
The last column lists all variables that are implicated by the partial assignment but are not obtained by unit propagation.

Our algorithm also prints all possible solutions for the given cnf. These are:
\begin{lstlisting}
[{1, 2, 3, 4, 5}, 
 {1, 2, -5, -4, -3}, 
 {1, 3, 5, -4, -2}, 
 {1, -5, -4, -3, -2}, 
 {2, 4, -5, -3, -1}, 
 {2, -5, -4, -3, -1},
 {3, 4, -2, -5, -1}, 
 {3, -2, -5, -4, -1}, 
 {4, -2, -5, -3, -1}, 
 {-2, -5, -4, -3, -1}]
\end{lstlisting}
\begin{longtable}{|c|c|c|c|c|c|c|}
\hline 1 & 2 & 3 & 4 & 5 & unit propagation & optimal? \\ \hline 
true & true & true & true & - & 5               &    \\ \hline 
true & true & true & - & true & 4               &    \\ \hline 
true & true & true & - & - & 4,5             &    \\ \hline 
true & true & false & false & - & -5              &    \\ \hline 
true & true & false & - & false & -4              &    \\ \hline 
true & true & false & - & - & -5,-4           &    \\ \hline 
true & true & - & true & true & 3               &    \\ \hline 
true & true & - & true & - & 3,5             &    \\ \hline 
true & true & - & false & false & -3              &    \\ \hline 
true & true & - & false & - & -5,-3           &    \\ \hline 
true & true & - & - & true & 3,4             &    \\ \hline 
true & true & - & - & false & -4,-3           &    \\ \hline 
true & true & - & - & - &                 &    \\ \hline 
true & false & true & false & - & 5               &    \\ \hline 
true & false & true & - & true & -4              &    \\ \hline 
true & false & true & - & - & -4,5            &    \\ \hline 
true & false & false & false & - & -5              &    \\ \hline 
true & false & false & - & false & -4              &    \\ \hline 
true & false & false & - & - & -5,-4           &    \\ \hline 
true & false & - & false & true & 3               &    \\ \hline 
true & false & - & false & false & -3              &    \\ \hline 
true & false & - & false & - &                 &    \\ \hline 
true & false & - & - & true & 3,-4            &    \\ \hline 
true & false & - & - & false & -4,-3           &    \\ \hline 
true & false & - & - & - &                 & -4 \\ \hline 
true & - & true & true & true & 2               &    \\ \hline 
true & - & true & true & - & 2,5             &    \\ \hline 
true & - & true & false & true & -2              &    \\ \hline 
true & - & true & false & - & 5,-2            &    \\ \hline 
true & - & true & - & true &                 &    \\ \hline 
true & - & true & - & - & 5               &    \\ \hline 
true & - & false & false & false &                 &    \\ \hline 
true & - & false & false & - & -5              &    \\ \hline 
true & - & false & - & false & -4              &    \\ \hline 
true & - & false & - & - & -5,-4           &    \\ \hline 
true & - & - & true & true & 2,3             &    \\ \hline 
true & - & - & true & - & 2,3,5           &    \\ \hline 
true & - & - & false & true & 3,-2            &    \\ \hline 
true & - & - & false & false & -3              &    \\ \hline 
true & - & - & false & - &                 &    \\ \hline 
true & - & - & - & true & 3               &    \\ \hline 
true & - & - & - & false & -4,-3           &    \\ \hline 
true & - & - & - & - &                 &    \\ \hline 
false & true & false & true & - & -5              &    \\ \hline 
false & true & false & false & - & -5              &    \\ \hline 
false & true & false & - & false &                 &    \\ \hline 
false & true & false & - & - & -5              &    \\ \hline 
false & true & - & true & false & -3              &    \\ \hline 
false & true & - & true & - & -5,-3           &    \\ \hline 
false & true & - & false & false & -3              &    \\ \hline 
false & true & - & false & - & -5,-3           &    \\ \hline 
false & true & - & - & false & -3              &    \\ \hline 
false & true & - & - & - & -5,-3           &    \\ \hline 
false & false & true & true & - & -5              &    \\ \hline 
false & false & true & false & - & -5              &    \\ \hline 
false & false & true & - & false &                 &    \\ \hline 
false & false & true & - & - & -5              &    \\ \hline 
false & false & false & true & - & -5              &    \\ \hline 
false & false & false & false & - & -5              &    \\ \hline 
false & false & false & - & false &                 &    \\ \hline 
false & false & false & - & - & -5              &    \\ \hline 
false & false & - & true & false &                 &    \\ \hline 
false & false & - & true & - & -5              &    \\ \hline 
false & false & - & false & false &                 &    \\ \hline 
false & false & - & false & - & -5              &    \\ \hline 
false & false & - & - & false &                 &    \\ \hline 
false & false & - & - & - & -5              &    \\ \hline 
false & - & true & true & false & -2              &    \\ \hline 
false & - & true & true & - & -5,-2           &    \\ \hline 
false & - & true & false & false & -2              &    \\ \hline 
false & - & true & false & - & -5,-2           &    \\ \hline 
false & - & true & - & false & -2              &    \\ \hline 
false & - & true & - & - & -5,-2           &    \\ \hline 
false & - & false & true & false &                 &    \\ \hline 
false & - & false & true & - & -5              &    \\ \hline 
false & - & false & false & false &                 &    \\ \hline 
false & - & false & false & - & -5              &    \\ \hline 
false & - & false & - & false &                 &    \\ \hline 
false & - & false & - & - & -5              &    \\ \hline 
false & - & - & true & false &                 &    \\ \hline 
false & - & - & true & - & -5              &    \\ \hline 
false & - & - & false & false &                 &    \\ \hline 
false & - & - & false & - & -5              &    \\ \hline 
false & - & - & - & false &                 &    \\ \hline 
false & - & - & - & - & -5              &    \\ \hline 
- & true & true & true & true & 1               &    \\ \hline 
- & true & true & true & - & 1,5             &    \\ \hline 
- & true & true & - & true & 1,4             &    \\ \hline 
- & true & true & - & - & 1,4,5           &    \\ \hline 
- & true & false & true & false & -1              &    \\ \hline 
- & true & false & true & - & -5,-1           &    \\ \hline 
- & true & false & false & false &                 &    \\ \hline 
- & true & false & false & - & -5              &    \\ \hline 
- & true & false & - & false &                 &    \\ \hline 
- & true & false & - & - & -5              &    \\ \hline 
- & true & - & true & true & 1,3             &    \\ \hline 
- & true & - & true & false & -3,-1           &    \\ \hline 
- & true & - & true & - &                 &    \\ \hline 
- & true & - & false & false & -3              &    \\ \hline 
- & true & - & false & - & -5,-3           &    \\ \hline 
- & true & - & - & true & 1,3,4           &    \\ \hline 
- & true & - & - & false &                 & -3 \\ \hline 
- & true & - & - & - &                 &    \\ \hline 
- & false & true & true & false & -1              &    \\ \hline 
- & false & true & true & - & -5,-1           &    \\ \hline 
- & false & true & false & true & 1               &    \\ \hline 
- & false & true & false & false & -1              &    \\ \hline 
- & false & true & false & - &                 &    \\ \hline 
- & false & true & - & true & 1,-4            &    \\ \hline 
- & false & true & - & false & -1              &    \\ \hline 
- & false & true & - & - &                 &    \\ \hline 
- & false & false & true & false & -1              &    \\ \hline 
- & false & false & true & - & -5,-1           &    \\ \hline 
- & false & false & false & false &                 &    \\ \hline 
- & false & false & false & - & -5              &    \\ \hline 
- & false & false & - & false &                 &    \\ \hline 
- & false & false & - & - & -5              &    \\ \hline 
- & false & - & true & false & -1              &    \\ \hline 
- & false & - & true & - & -5,-1           &    \\ \hline 
- & false & - & false & true & 1,3             &    \\ \hline 
- & false & - & false & false &                 &    \\ \hline 
- & false & - & false & - &                 &    \\ \hline 
- & false & - & - & true & 1,3,-4          &    \\ \hline 
- & false & - & - & false &                 &    \\ \hline 
- & false & - & - & - &                 &    \\ \hline 
- & - & true & true & true & 1,2             &    \\ \hline 
- & - & true & true & false & -2,-1           &    \\ \hline 
- & - & true & true & - &                 &    \\ \hline 
- & - & true & false & true & 1,-2            &    \\ \hline 
- & - & true & false & false & -1,-2           &    \\ \hline 
- & - & true & false & - & -2              &    \\ \hline 
- & - & true & - & true & 1               &    \\ \hline 
- & - & true & - & false & -2,-1           &    \\ \hline 
- & - & true & - & - &                 &    \\ \hline 
- & - & false & true & false & -1              &    \\ \hline 
- & - & false & true & - & -5,-1           &    \\ \hline 
- & - & false & false & false &                 &    \\ \hline 
- & - & false & false & - & -5              &    \\ \hline 
- & - & false & - & false &                 &    \\ \hline 
- & - & false & - & - & -5              &    \\ \hline 
- & - & - & true & true & 1,2,3           &    \\ \hline 
- & - & - & true & false & -1              &    \\ \hline 
- & - & - & true & - &                 &    \\ \hline 
- & - & - & false & true & 1,3,-2          &    \\ \hline 
- & - & - & false & false &                 &    \\ \hline 
- & - & - & false & - &                 &    \\ \hline 
- & - & - & - & true & 1,3             &    \\ \hline 
- & - & - & - & false &                 &    \\ \hline 
- & - & - & - & - &                 &    \\ \hline 
\end{longtable}


From the table we obtain two partial valuations for which the cnf does not optmially propagate. ${x_1 = true, x_2=false}$ and ${x_2 = true, x_5=false}$.

The following algorithm was used.

First, we input the cnf and the amount of variables into \verb|optimal_table|.

Then we compute all possible solutions of the cnf by brute forcing all complete valuations.
We check if the cnf is fulfilled by substracting the set of variables (Example: \verb|variables = {1,2,-3,4,-5}|) of each cnf clause. If all the cnf clauses are empty by the end the cnf is fulfilled.

After that we loop over all possible valuations. A partial valuation for 5 variables can be represented as the cross product $\{-1,0,1\}^{5}$. For example $(0,-1,0,0,1)$ means $\{x_2=false, x_5=true\}$
In this loop we get all derived variables via recursive unit propagation. We then check for all supersets of our current partial valuations that are a solution to the cnf.
If any variable is implicated by all these supersets but not gained from unit propagation the formula is not propagating optimally.

Note: The code might be kind of hard to read since it uses may set operations like \verb|A - B|.


\lstinputlisting{propagation.py}


\section*{Exercise 2: Tseitin Encoding}

We built our tesitin encoded formula this way:
\begin{equation*} \label{eq1}
\begin{split}
    A &=\psi \land ((x\land z)\lor y)\land (((y\oplus a)\lor z)\oplus \neg y)\\
    &= t_{1}\land (t_{1}\leftrightarrow (\psi \land((x\land z)\lor y)\land (((y\oplus a)\lor z)\oplus \neg y))\\
    &= t_{1}\land (t_{1}\leftrightarrow (t_{2}\land \psi))\land(t_{2}\leftrightarrow(((x\land z)\lor y)\land(((y\oplus a)\lor z)\oplus \neg y))\\
    &= t_{1}\land (t_{1}\leftrightarrow (t_{2}\land \psi))\land(t_{2}\leftrightarrow(t_{3}\land t_{4}))\land(t_{3}\leftrightarrow((x\land z)\lor y))\land(t_{4}\leftrightarrow(((y\oplus a)\lor z)\oplus \neg y))\\
    &= t_{1}\land (t_{1}\leftrightarrow (t_{2}\land \psi))\land(t_{2}\leftrightarrow(t_{3}\land t_{4}))\land(t_{3}\leftrightarrow(t_{5}\lor y))\land(t_{5}\leftrightarrow(x\land z))\land(t_{4}\leftrightarrow(t_{6}\oplus \neg y))\\
    &\land(t_{6}\leftrightarrow((y\oplus a)\lor z))\\
    &= t_{1}\land (t_{1}\leftrightarrow (t_{2}\land \psi))\land(t_{2}\leftrightarrow(t_{3}\land t_{4}))\land(t_{3}\leftrightarrow(t_{5}\lor y))\land(t_{5}\leftrightarrow(x\land z))\land(t_{4}\leftrightarrow(t_{6}\oplus \neg y))\\
    &\land(t_{6}\leftrightarrow(t_{7}\lor z))\land(t_{7}\leftrightarrow(y\oplus a))
\end{split}
\end{equation*}

After transforming each sub formula into cnf clauses we get:
\begin{align*}
t_{1}&   &  \neg t_{3}\lor \neg &t_{4}\lor t_{2} &  t_{5}\lor y &\lor\neg t_{3} & \neg x\lor \neg &z\lor t_{5}\\
\neg t_{2}\lor  \neg &\psi \lor t_{1}   &  t_{3}\lor &\neg t_{2}   & \neg t_{5}&\lor t_{3} & x\lor &\neg t_{5}\\
t_{2}\lor &\neg t_{1}   &  t_{4}\lor &\neg t_{2} &  \neg y&\lor t_{3} & z\lor &\neg t_{5}\\
\psi \lor &\neg t_{1}\\
\\
\neg t_{6}\lor y \lor & \neg t_{4} & t_{7}\lor z& \lor \neg t_{6} & \neg y \lor \neg &a \lor \neg t_{7}\\
t_{6}\lor \neg y\lor &\neg t_{4} & \neg t_{7}&\lor t_{6} & y \lor a &\lor \neg t_{7}\\
t_{6}\lor y& \lor t_{4} & \neg z&\lor t_{6} & y \lor \neg &a \lor t_{7}\\
\neg t_{6}\lor \neg &y \lor t_{4} &    &     & \neg y \lor  &a \lor t_{7}
\end{align*}

Finally we can remove some clauses after performing the polarity check.
\begin{align*}
t_{1}&   &  \textcolor{red}{\neg t_{3}\lor \neg} &\textcolor{red}{t_{4}\lor t_{2}} &  t_{5}\lor y &\lor\neg t_{3} & \textcolor{red}{\neg x\lor \neg }&\textcolor{red}{z\lor t_{5}}\\
\textcolor{red}{\neg t_{2}\lor  \neg }&\textcolor{red}{\psi \lor t_{1}}   &  t_{3}\lor &\neg t_{2}   & \textcolor{red}{\neg t_{5}}&\textcolor{red}{\lor t_{3}} & x\lor &\neg t_{5}\\
t_{2}\lor &\neg t_{1}   &  t_{4}\lor &\neg t_{2} &  \textcolor{red}{\neg y}&\textcolor{red}{\lor t_{3}} & z\lor &\neg t_{5}\\
\psi \lor &\neg t_{1}\\
\\
\neg t_{6}\lor y\lor & \neg t_{4} & t_{7}\lor z& \lor \neg t_{6} & \neg y \lor \neg &a \lor \neg t_{7}\\
t_{6}\lor \neg y\lor& \neg t_{4} & \textcolor{red}{\neg t_{7}}&\textcolor{red}{\lor t_{6}} & y \lor a &\lor \neg t_{7}\\
\textcolor{red}{t_{6}\lor y}& \textcolor{red}{\lor t_{4}} & \textcolor{red}{\neg z}&\textcolor{red}{\lor t_{6}} & \textcolor{red}{y \lor \neg} &\textcolor{red}{a \lor t_{7}}\\
\textcolor{red}{\neg t_{6}\lor \neg} &\textcolor{red}{y \lor t_{4}} &    &     & \textcolor{red}{\neg y \lor}  &\textcolor{red}{a \lor t_{7}}
\end{align*}
Thus, we were able to reduce the overall number of clauses from 24 to 13.
\end{document}
