\documentclass{base}
% Dateikodierung ist utf8
\usepackage[utf8]{inputenc}
\usepackage{url}
\usepackage[export]{adjustbox}
\usepackage{amsmath}
\usepackage{listings}
\usepackage{tikz}
\usepackage{tabularx}
\usepackage{color,colortbl}
\usepackage{ulem}
\usepackage{pdfpages}
\usepackage{ wasysym }
\usepackage{ booktabs }
\usepackage{lscape}
\usepackage{multicol}
\usepackage{longtable}

\begin{document}

\Abgabeblatt{Assignment 5}{21.5.2018}{????}{????}{Yannis Rohloff (yannis@uni-bremen.de)}{Meng Liu(lium@uni-bremen.de)}{Islam Abushanab(is\_ab@uni-bremen.de)}

\lstset{
    language=Python,
    basicstyle=\ttfamily\small,
    aboveskip={1.0\baselineskip},
    belowskip={1.0\baselineskip},
    columns=fixed,
    extendedchars=true,
    breaklines=true,
    tabsize=4,
    prebreak=\raisebox{0ex}[0ex][0ex]{\ensuremath{\hookleftarrow}},
    frame=lines,
    showtabs=false,
    showspaces=false,
    showstringspaces=false,
    keywordstyle=\color[rgb]{0.627,0.126,0.941},
    commentstyle=\color[rgb]{0.133,0.545,0.133},
    stringstyle=\color[rgb]{01,0,0},
    numbers=left,
    numberstyle=\small,
    stepnumber=1,
    numbersep=10pt,
    captionpos=t,
    escapeinside={\%*}{*)}
}


\section*{Exercise 1: }

We already have $\phi$, which is a ladder encoding. We'll use that later.\\
An encoded v is odd if for an odd i the variable $y_i$ is true but $y_{i-1}$ is not, since that would encode that $v=i$.
To deny such valuations we set
$$\psi_n = \bigwedge_{\substack{i \in \{0,\dots,n-1\} \\ i\text{ is odd}}}\ (\neg y_i \lor y_{i-1})$$
Read it as: ``if an odd $y_i$ is true, its predecessor(s) must be true too.''

\textbf{Consider n is even:} \\
We know $|\psi_n| = \frac{n}{2}$. \\
For example: In case $n=2$ we have 1 clause stating that if $y_1$ is true, $y_0$ must be true, too, which is correct, since otherwise $v=1$ and that is would be odd. \\
We got both formulas via the observation shown in the task itself.

\textbf{Consider n is odd:} \\
We observe that $|\psi_n| = \frac{n-1}{2}$. \\

So:
$$f(n) = \begin{cases} 
        \frac{n}{2} & \text{if n is even} \\
        \frac{n-1}{2} & \text{if n is odd} \\
    \end{cases}
$$




\section*{Exercise 2:}
As we can see the clauses that we get from the constraint are:
	\begin{equation}
		\psi = \bigwedge\limits_{0\leq i \leq n-3} (\neg y_{i}\lor y_{i+1})
	\end{equation}
    If we know at least one of the assignment to a literal, we can perform unit propagation.
    If we set $y_i=true$ we get from unit propagation, that all $y_j$ with $j>i$ are true too. So the increasing direction can entirely pe propagated. Looking in the other direction we start with the clause $(\neg y_{i-1} \lor y_i)$ if $i>0$.
    This clause is already satisfied due to $y_i=true$, meaning $y_{i-1}$ can be either true or false.

Setting $y_i=false$ reverses this analysis. All leading clauses and literals can be propagated but not the succesors of $y_i$, since $(y_i \lor y_{i+1})$ does not become a unit-clause.
    
As soon as we learn a pair of variables $y_i, y_{i+1}$ with $y_i=false$ and $y_{i+1}=true$ we can optimally propagate all other literals.









\end{document}
