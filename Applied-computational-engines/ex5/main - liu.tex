\documentclass{base}
% Dateikodierung ist utf8
\usepackage[utf8]{inputenc}
\usepackage{url}
\usepackage[export]{adjustbox}
\usepackage{amsmath}
\usepackage{listings}
\usepackage{tikz}
\usepackage{tabularx}
\usepackage{color,colortbl}
\usepackage{ulem}
\usepackage{pdfpages}
\usepackage{ wasysym }
\usepackage{ booktabs }
\usepackage{lscape}
\usepackage{multicol}
\usepackage{longtable}

\begin{document}

\Abgabeblatt{Assignment 5}{21.5.2018}{????}{????}{Yannis Rohloff (yannis@uni-bremen.de)}{Meng Liu(lium@uni-bremen.de)}{}

\lstset{
    language=Python,
    basicstyle=\ttfamily\small,
    aboveskip={1.0\baselineskip},
    belowskip={1.0\baselineskip},
    columns=fixed,
    extendedchars=true,
    breaklines=true,
    tabsize=4,
    prebreak=\raisebox{0ex}[0ex][0ex]{\ensuremath{\hookleftarrow}},
    frame=lines,
    showtabs=false,
    showspaces=false,
    showstringspaces=false,
    keywordstyle=\color[rgb]{0.627,0.126,0.941},
    commentstyle=\color[rgb]{0.133,0.545,0.133},
    stringstyle=\color[rgb]{01,0,0},
    numbers=left,
    numberstyle=\small,
    stepnumber=1,
    numbersep=10pt,
    captionpos=t,
    escapeinside={\%*}{*)}
}


\section*{Exercise 1: }
The idea about adding the clause is to make sure that if $y_{2i}$ is \textbf{false}, $y_{2i+1}$ cannot be \textbf{true}.
	\begin{enumerate}
		\item $n$ is even:
		\begin{equation}
			\bigwedge\limits_{0\leq i \leq \frac{n-2}{2}} (y_{2i}\lor \neg y_{2i+1})
		\end{equation}
		
		\item $n$ is odd:
		\begin{equation}
			\bigwedge\limits_{0\leq i \leq \frac{n-3}{2}} (y_{2i}\lor \neg y_{2i+1})
		\end{equation}
	\end{enumerate}
	The additional clauses will be:
	\[ f(n) = 
	\begin{cases} 
	\frac{n}{2} & \text{if $n$ is even} \\
	\frac{n-1}{2} & \text{if $n$ is odd}
	\end{cases}
	\]
	
	The $\varphi$ has $4n-1$ clauses, adding the new clauses above, the total will be:
		\[ f(n) = 
		\begin{cases} 
		\frac{9n}{2} -1 & \text{if $n$ is even} \\
		\frac{9n}{2} - \frac{3}{2} & \text{if $n$ is odd}
		\end{cases}
		\]

\section*{Exercise 2:}
As we can see from the clauses come from the constraint:
	\begin{equation}
		\psi = \bigwedge\limits_{0\leq i \leq n-3} (\neg y_{i}\lor y_{i+1})
	\end{equation}
If we know at least one of the assignment to a literal, after cross out all the satisfied clauses and falsified literals, the rest will be unit literal in a clause. In this case, the unit propagation can derive the value to all the variables.
\end{document}
