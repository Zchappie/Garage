\documentclass{base}
% Dateikodierung ist utf8
\usepackage[utf8]{inputenc}
\usepackage{url}
\usepackage[export]{adjustbox}
\usepackage{amsmath}
\usepackage{listings}
\usepackage{tikz}
\usepackage{tabularx}
\usepackage{ulem}
\usepackage{pdfpages}
\usepackage{ wasysym }
\usepackage{ booktabs }
\usepackage{lscape}
\usepackage{multicol}
\usepackage{longtable}
\usepackage{flexisym}
\usepackage[]{algorithm2e}

\usepackage{xcolor}
\begin{document}

\Abgabeblatt{Assignment 11}{2.7.2018}{????}{????}{Yannis Rohloff (yannis@uni-bremen.de)}{Meng Liu(lium@uni-bremen.de)}{Islam Abushanab (is\_ab@uni-bremen.de)}

\lstset{
    language=Python,
    basicstyle=\ttfamily\small,
    aboveskip={1.0\baselineskip},
    belowskip={1.0\baselineskip},
    columns=fixed,
    extendedchars=true,
    breaklines=true,
    tabsize=4,
    prebreak=\raisebox{0ex}[0ex][0ex]{\ensuremath{\hookleftarrow}},
    frame=lines,
    showtabs=false,
    showspaces=false,
    showstringspaces=false,
    keywordstyle=\color[rgb]{0.627,0.126,0.941},
    commentstyle=\color[rgb]{0.133,0.545,0.133},
    stringstyle=\color[rgb]{01,0,0},
    numbers=left,
    numberstyle=\small,
    stepnumber=1,
    numbersep=10pt,
    captionpos=t,
    escapeinside={\%*}{*)}
}



\section*{SAT+LP}
Give names to the ground terms first.

\begin{tabular}[]{@{\hspace{-1.5cm}}ccccccccccc}
$(x \leq 5 \lor y \geq 2)  $&$\land $ & $(x \geq 4 \lor y \leq 2) $&$ \land $        & $(x+y \geq 12)$&$ \land $  & $ (x+y \leq 14)$&$ \land $ & $ (3y-4x \leq 0)$&$\land $ & $ (3x-4y \leq 0)$ \\
$(a \lor b)$&$ \land         $        & $ (c \lor d) $&$ \land                     $ & $ (e)$&$ \land           $ & $ (f)$&$ \land           $ & $ (g)$&$\land            $ & $ (h)$ \\
\end{tabular}

The first Alrogithm divides into multiple steps.
(1) Unit Propagation, (2) Falsification of clauses, (3) Solve LP Problem of the true constraints and (4) Recursing with a new variable.

We will now recurse through these steps.


%Since it is hard to recognize the steps of recursion when written in a linear text we first considered only the steps (1),(2) and (4) and assumed, that no full set of terms yields a possible solution with LP.
%We built the following table of variable combinations.


We write $a=true$ as $a$ and $a=false$ as $\overline a$.
The first four variables $e,f,g,h$ are always true by unit propagation in the first recursion step.

The third column is step (3). This will only be executed if all terms are either included or excluded from the LP instance.
The concrete lp-instances are listed below the table

\begin{tabular}[]{|l|l|l|}
\hline
\textbf{A}                              & \textbf{(1) and (2)} & \textbf{(3)}\\ \hline
$e,f,g,h$                               & unit propagation & \\ \hline
$e,f,g,h,\overline a$                   &                  & \\ \hline
$e,f,g,h,\overline a,b$                 & unit propagation & \\ \hline
$e,f,g,h,\overline a,b, \overline c$    &                  & \\ \hline
$e,f,g,h,\overline a,b, \overline c, d$ & unit propagation & infeasable\\ \hline
$e,f,g,h,\overline a,b, c, $            &                  & \\ \hline
$e,f,g,h,\overline a,b, c, \overline d$ &                  & solution found! x=6,y=6\\ \hline
$e,f,g,h,\overline a,b, c, d$           &                  & -\\ \hline
$e,f,g,h,a,$                            &                  & -\\ \hline
$e,f,g,h,a,\overline b, \overline c$    &                  & -\\ \hline
$e,f,g,h,a,\overline b, \overline c, d$ & unit propagation & -\\ \hline
$e,f,g,h,a,\overline b, c$              &                  & -\\ \hline
$e,f,g,h,a,\overline b, c, \overline d$ &                  & -\\ \hline
$e,f,g,h,a,\overline b, c, d$           &                  & -\\ \hline
$e,f,g,h,a, b$                          &                  & -\\ \hline
$e,f,g,h,a, b, \overline c$             &                  & -\\ \hline
$e,f,g,h,a, b, \overline c, d$          & unit propagation & -\\ \hline
$e,f,g,h,a, b, c$                       &                  & -\\ \hline
$e,f,g,h,a, b, c, \overline d$          &                  & -\\ \hline
$e,f,g,h,a, b, c, d$                    &                  & -\\ \hline
\end{tabular}


\section*{Uniterpreted Functions}

There are the following sub-terms:
$$x,y,f(x), f(y), f(f(x)),g(x,x), g(x,y), g(y,x), g(f(y),y), g(f(y),f(y)), g(f(x), f(x)), g(x, g(x,y)) $$
After merging them according to step 3 we get the following groups.

\renewcommand{\c}[2][c]{%
  \begin{tabular}[#1]{@{}c@{}}#2\end{tabular}}
\begin{tabular}[]{|l|l|l|l|l|l|l|l|}
\hline
\c{$x$\\$f(x)$} & 
\c{$f(f(x)) $\\$ g(x,x)$} & 
\c{$g(f(y), y) $\\$ g(f(y),f(y))$} &
\c{$ f(y) $\\$ g(f(x), f(x)) $} & 
\c{$ g(x,g(x,y)) $\\$ g(y,x) $} &
\c{$y$} & 
\c{$g(x,y)$} 
\\ \hline
\end{tabular}

We match: $f(f(x)) = f(f(x)) $

\begin{tabular}[]{|l|l|l|l|l|l|l|l|}
\hline
\c{$x$\\$f(x)$ \\ $f(f(x)) $\\$ g(x,x)$} & 
\c{$g(f(y), y) $\\$ g(f(y),f(y))$} &
\c{$ f(y) $\\$ g(f(x), f(x)) $} & 
\c{$ g(x,g(x,y)) $\\$ g(y,x) $} &
\c{$y$} & 
\c{$g(x,y)$} 
\\ \hline
\end{tabular}

We match: $g(x,x)) = g(f(x), f(x)) $

\begin{tabular}[]{|l|l|l|l|l|l|l|l|}
\hline
\c{$x$\\$f(x)$ \\ $f(f(x)) $\\$ g(x,x)$ \\ $ f(y) $\\$ g(f(x), f(x)) $} & 
\c{$g(f(y), y) $\\$ g(f(y),f(y))$} &
\c{$ g(x,g(x,y)) $\\$ g(y,x) $} &
\c{$y$} & 
\c{$g(x,y)$} 
\\ \hline
\end{tabular}

We match: $g(x,x) = g(f(y), f(y)) $

\begin{tabular}[]{|l|l|l|l|l|l|l|l|}
\hline
\c{$x$\\$f(x)$ \\ $f(f(x)) $\\$ g(x,x)$ \\ $ f(y) $\\$ g(f(x), f(x)) $ \\ $g(f(y), y) $\\$ g(f(y),f(y))$} &
\c{$ g(x,g(x,y)) $\\$ g(y,x) $} &
\c{$y$} & 
\c{$g(x,y)$} 
\\ \hline
\end{tabular}

No matter if we could merge again. In the end we need to return ``inconsistent'', because of a conflict:

$g(x,x)$ and $g(f(y), y)$ are in the same equivalence class, but $g(x,x) \neq g(f(y), y)$ is part of the set of ground terms.
\end{document}
